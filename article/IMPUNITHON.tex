%\FreeChap{picture 840x390}{Name of chap.}
%\FreeChap{six}{Free Software in Tunisian Government}

\newpage
\FreeChap{IMPUNITHON.jpg}{IMPUNITHON \normalsize Freeways  remporte la 2\`eme place .. Et l'ISI source d'inspiration}
%\NewsItem{\textcolor[rgb]{0.55,0,0}{\huge Simple and Fast Multimedia Library }}
\vspace*{-30mm}

\begin{multicols}{2}
Le club Freeways ISI a particip\'e lundi 29 d\'ecembre a "IMPUNITHON" co-organis\'e par I-WATCH TUNISIA et GDG Tunis et qui a eu lieu \`a ESPRIT. IMPUNITHON est un hackathon qui dure 24 heures pendant les-quelles 20 \'equipes de jeunes \'etudiants et professionnels doivent d\'evelopper un produit fini qui touche les trois th\'emes propos\'es par les organisateurs :\\
1- Prise de d\'ecision participative \\
2- Mettre fin \`a l'impunit\'e \\
3- Responsabilisation\\

\begin{center}
\textbf{La participation de Freeways}\\
\end{center}

Notre club a \'et\'e pr\'esent \`a cet \'ev\`enement apr\'es avoir \'et\'e s\'electionn\'e parmi les 20 \'equipes qui vont participer, avec une \'euipe compos\'ee de 3 membres :\textbf{ Mohamed Manai} (1\'ere ann\'ee Master, \textbf{Sabri Bahrini} (2\'eme ann\'ee du cycle Ing.) et \textbf{Emir Ben Khadda} (1\'ere ann\'ee Licence), et qui participe \`a son premier hackathon).
Une jeune \'equipe (avec une moyenne d'ages de 21 ans) qui a r\'eussi \`a gagner la 2\'eme place avec son projet "\textbf{Fac-ki!}" qui vise a aider les \'etudiants \`a prendre des d\'ecisions, suivre les promesses de l'administration et les repr\'esentants dans les conseils scientifiques et avoir aussi un acc\`es aux donn\'ees et aux documents concernant l'\'etablissement \`a la fa\c con OpenData.\\

\begin{center}
\textbf{ISI source d'inspiration}\\
\end{center}

Les th\`emes propos\'es \'etaient hors le domaine et le cercle de confort des participants, c'est pour cela certaines \'equipes ont eu du mal \`a analyser les probl\`emes avant de penser \`a la solution. Les projets finaux circulaient au tour des \'etablissements publiques et comment aider les citoyens \`a faire entendre leurs voix et signaler les probl\`emes aux responsables, mais d'une fa\c con g\'en\'erale. L'id\'ee initiale de notre \'equipe \'etait de s'int\'eresser \`a un seul \'etablissement, l'universit\'e en particulier et essayer de proposer des solutions. Trouver les probl\`emes reli\'ees \`a l'universit\'e et \`a notre institut sp\'ecialement n'\'etait pas quelque chose de difficile puisque on est parmi les plus proches \`a analyser ces probl\`emes et ce qui manque  en essayant bien sûr de les corriger.\\

\begin{center}
\textbf{Les ISI'tiens et IMPUNITHON }\\
\end{center}

La pr\'esence des \'etudiants de l'ISI \'etait int\'eressante, avec 2 \'equipes qui repr\'esentent Freeways Club et ISI Google Club (qui ont fait un tr\`es grand boulot avec leur premi\`ere participation) mais aussi avec des \'etudiants qui ont particip\'e \`a l'organisation et avec d'autres \'equipes, dont une qui a gagn\'e la 4\`eme place. Un bravo \`a tous les ISI'tiens et les ISI'tiennes qui ont bien repr\'esent\'e l'ISI avec cette participation, on s'attend plus de succ\`es pour les prochaines \'ev\`enements.\\

Un dernier mot pour les \'etudiants de l'ISI
\`a l'ISI on a des \'etudiants brillants, mais il reste des talents \`a d\'ecouvrir. J'esp\`ere que cet article sera une motivation pour vous encourager \`a bouger et faire autre(s) activit\'e(s) en parall\`ele avec vos \'etudes. Rejoindre les clubs, que ce soit le club Freeways ou les autres, ne peut que vous aider \`a d\'evelopper vos connaissances et \`a sentir le plaisir de faire une activit\'e que vous aimez et de changer quelque chose dans ce monde.\\

Enfin, un grand bravo aux jeunes \'etudiants - d\'eveloppeurs qui montrent une autre fois leurs importance comme des memebres de la soci\'et\'e civile et leurs comp\'etences pour aider \`a d\'evelopper notre pays.


\end{multicols}

\NewsAuthorEx{\small Mohamed}{MANAI}{
\vspace*{-6mm}
\begin{flushright}
\scriptsize \textcolor{GreenTea}{\Estatically~} Designer, Freeways ISI\\
\end{flushright}
}