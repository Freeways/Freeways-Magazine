%\FreeChap{picture 840x390}{Name of chap.}
%\FreeChap{six}{Free Software in Tunisian Government}

\newpage
\vspace{3cm}
 
\NewsItem{\textcolor[rgb]{0.55,0,0}{\huge Simple and Fast Multimedia Library }}

\NewsAuthorEx{\small Houssem}{JELLITI}{
\vspace*{-6mm}
\begin{flushright}
\scriptsize \textcolor{GreenTea}{\Estatically~} Member, Freeways ISI\\
\end{flushright}
}

\parag{Simple and Fast Multimedia Library (SFML) is a cross-platform software development library designed to provide a simple interface to various multimedia components in computers. However, feel free to call it a 'rendering' engine as well.}
\parag{Written in C++, with bindings available for C, Java, Python, Ruby, .NET, Go (and others). SFML is Free and open-source software provided under the terms of the zlib/png license. It is available on Windows, Linux, OS X and FreeBSD.}
\parag{Back in 2006, Laurent Gomila (Software Engineer - France) started working on an API which can be an alternative to SDL with Object-Oriented style using C++, if you don't know what SDL is please visit: libsdl.org.}
\parag{Laurent once wrote: "When I started to write this library in 2006, I couldn't imagine that it would become so much popular. Around 100,000 visitors per month, 100 new forum posts everyday \dots{}
this is huge!" - SFML Game Development Book.}
\parag{In 2007, SFML 1.0 appeared.Since that time, it was under development until it reached the 2.0 version during 2013. However, the current version of the API is 2.2 (December 2014) .The 2.2 version do have support for the Android/iOS devices, and this is a good feature to take the level more up . As the developers think and as mentioned in the official web site,it is recommended ther you use the 2.2 version because it is a stable release  with the latest features and bugfixes. As such it will help you to avoid headaches because other versions such as 1.6 are not maintained anymore, containing  few critical bugs and lacking a lot of useful features. Let's take a look at the software architecture:}
\parag{System:\\
Vector and Unicode string classes, portable threading and timer facilities.}
\parag{Window:\\
Window and input devices management including support for joysticks, openGL context management.}
\parag{Graphics:\\
Hardware-accelerated 2D graphics including sprites, polygons and text rendering.}
\parag{Audio:\\
Hardware-accelerated spatialised audio playback and recording.}
\parag{Network:\\
TCP and UDP sockets, data encapsulation facilities, HTTP and FTP classes.}
\parag{SFML handles window creation and input as well as the creation and management of OpenGL contexts. It also provides a graphics module for simple hardware-accelerated 2D graphics which includes text rendering, an audio module that utilizes OpenAL and a networking module for basic TCP and UDP communication. However, the Graphics module is one of the main features of SFML ; developers who are only interested in creating an environment to program directly in OpenGL can do so by using the Window module on its own without the Graphics module. Similarly, the other modules can be used independently of one another as well with the exception of the system module which is used by all other modules.}
\parag{Whether you have a question about SFML's API, you experience an odd behavior with SFML or you have a 
feature request, you'll certainly find help, answers or feedback on the official forum: sfml-dev.org. The SFML community contains code snippets, tutorials and other community-contributed content ; you can learn how to use the API in the learning section. It is a good destination for SFML fans, Compared to older libraries such as SDL and Allegro, It is growing and promising .The only lack for me is the name \dots{} pretty long huh ??}