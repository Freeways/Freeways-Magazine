\chapter{Free Software in Tunisian Government}
\paragraph*{There are different reasons that favor the use of Free Software instead of the proprietary software used by governments and particulars. The main ones are cost and security.
Tunisian Ministries like Defense or Interior use proprietary products, mainly Microsoft's, which are known to contain built-in back doors used for spying on their users. As mentioned in the Free Software Foundation's website : "Microsoft Windows provides back doors for the NSA."}
\paragraph*{As Deputy Sayida Ounissi declared in the Parlimenthon 2015 event, Tunisian deputies use Facebook to communicate and discuss governmental issues which are supposedly confidential, these information are hidden from regular people but not from Facebook administration, which was accused several times of selling and providing information to interested parties which include the NSA. In their 7th of June 2013's copy, The Guardian posted an article about multiple companies giving direct access to NSA. "The National Security Agency has obtained direct access to the systems of Google, Facebook, Apple and other US internet giants, according to a top secret document obtained by the Guardian."}
\paragraph*{On the other hand The United States Department of Defense uses GNU/Linux.}
\paragraph*{Poor countries like Tunisia can benefit from the use of Free Software, not just to improve security but also to decrease spending.
France's national police upgraded 85,500 PC's to Ubuntu Desktop Edition saving 2 million euros a year in licence fees alone.}
\paragraph*{The question remains the same, if Free Software and Open Source offers better quality, higher reliability, and lower cost, why our government (or any government) insists on partnering with Microsoft ?}




Sources:
https://www.fsf.org/news/reform-corporate-surveillance
http://www.theguardian.com/world/2013/jun/06/us-tech-giants-nsa-data
http://www.ubuntu.com/products/casestudies/french-national-police-force-saves-2-million-year-ubuntu