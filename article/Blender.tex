%\FreeChap{picture 840x390}{Name of chap.}
%\FreeChap{six}{Free Software in Tunisian Government}

\newpage
\vspace{3cm}
 
\NewsItem{\textcolor[rgb]{0.55,0,0}{\huge Blender}}
\NewsAuthorEn{\small Hedi}{BELAIBA}

\paragraph*{Depuis peu Blender est enfin devenu un outil open source! Il est temps d'explorer cet outil fantastique. Vous allez constater qu'il offre des tas de possibilit\'es pour cr\'eer \`a peu pr\`es tout ce que vous souhaitez! Cependant, si vous n'avez jamais utilis\'eBlender ou tout autre outil de conception graphique en 3D, vous serez probablement accabl\'eet confus face \`a la profusion des boutons et d'options. Commençons donc par le tout d\'ebut.}
\paragraph*{Blender a \'et\'econ\c u en d\'ecembre 1993 avant de devenir un produit utilisable en Ao\^ut 1994 en tant qu'application int\'egr\'ee permettant la cr\'eation d'une gamme vari\'ee de productions 2D et 3D. Blender propose un large \'eventail de fonctionnalit\'es de modelage, de texturisation, d'\'eclairage, d'animation et de post-traitement vid\'eo dans un seul progiciel. Gr\^ace \`a son architecture libre, il procure une interop\'erabilit\'emultiplate-forme, une extensibilit\'e, une discr\'etion remarquable et un workflow (mod\'elisation de la gestion des processus op\'erationnels) \'etroitement int\'egr\'e. Blender est l'une des applications graphiques 3D Open Source les plus populaires dans le monde.}
\paragraph*{Initialement d\'evelopp\'epar la soci\'et\'eNot a Number (NaN), Blender est devenu un « logiciel libre », dont le code source est disponible sous licence GNU GPL. Ce d\'eveloppement est aujourd'hui coordonn\'epar la Fondation Blender, bas\'ee au Pays-Bas.}
\paragraph*{Destin\'eaux professionnels des m\'edias et artistes du monde entier, Blender peut \^e tre utilis\'epour cr\'eer des visualisations 3D, des images fixes ainsi que des vid\'eos de qualit\'ecin\'ematographique, tandis que l'incorporation d'un moteur 3D en temps r\'eel permet la cr\'eation des productions 3D interactives en lecture autonome.}
\paragraph*{Entre 2008 et 2010, des portions-cl\'ede Blender ont \'et\'er\'e\'ecrites pour am\'eliorer ses fonctionnalit\'es, son workflow et son interface.
De nombreux outils et techniques disponibles avec Blender vous aide \`a cr\'eer facilement vos mod\`eles ; il existe par exemple de nombreuses formes "basiques" (souvent appel\'ees aussi "primitives") que vous pouvez utiliser et "remodeler"...}
\paragraph*{Bien, assez de "th\'eorie", il est temps de mettre nos connaissances en pratique et de se plonger dans Blender, voici quelques exemples de nos travaux :}