%\FreeChap{picture 840x390}{Name of chap.}
\FreeChap{Kinectmedium.jpeg}{Kinect et Open Source: \normalsize Quel avenir ensemble?}
%\FreeChap{KinectforWind.jpeg}{Kinect et Open Source: \normalsize Quel avenir ensemble?}
%\FreeChap{D-kinect10.jpg}{Kinect et Open Source: \small Quel avenir ensemble?}

%\NewsItem{\textcolor[rgb]{0.55,0,0}{\huge Kinect et Open Source: Quel avenir ensemble?}}

\vspace*{-60mm}


\parag{\hspace*{6mm} En \'evoquant le mot high-tech, il est impossible de ne pas parler du p\'eriph\'erique Kinect (Projet Natal lors de la phase de d\'eveloppement) et de sa technologie novatrice. Sortie le 4 Novembre 2010 (date de sortie aux \`Etat-Unis) par Microsoft pour sa console Xbox 360 puis ensuite pour la Xbox One et Windows, Kinect est un p\'eriph\'erique de d\'etection du mouvement et de reconnaissance vocale et faciale permettant \`a l'utilisateur d'interagir avec la console ou l'ordinateur sans avoir recours \`a un contr\^oleur et cela uniquement avec des gestes et des commandes vocales, \`a travers ce qu'on appelle une Natural User Interface (NUI), ou interface utilisateur naturelle.}

\parag{Ce concept appara\^it extr\^emement prometteur, et c'est exactement ce que plusieurs personnes se sont dit : le mois m\^eme du lancement du gadget pour la Xbox 360, Adafruit Industries a lanc\'e un concours pour la cr\'eation d'un pilote open-source pour la Kinect dont le vainqueur, annonc\'e le 10 Novembre, a \'et\'e H\'ector Mart\`in Cantero, un hacker qui a pu produire un pilote Linux capable de capturer le flux vid\'eo RGB de la cam\'era du p\'eriph\'erique, mais aussi le flux monochrome g\'en\'er\'e par la profondeur de champ, apr\`es avoir p\'en\'etrer les protections mat\'erielles et logicielles mises en place par Microsoft. Suite \`a ce succ\`es et encore une fois le m\^eme mois, un ing\'enieur de Google d\'enomm\'e Matt Cutts a lui aussi organis\'e et financ\'e un concours pour le d\'eveloppement des applications pour Kinect pour rendre cette derni\`ere plus compatible avec le syst\`eme d'exploitation Linux. Il faut dire qu'ils ne perdent pas de temps ces gens-l\`a !}
\textcolor[rgb]{0.55,0,0}{OpenNI}

\parag{Le monde open source a accueilli \`a bras ouverts ces initiative, et une organisation nomm\'ee OpenNI a presque vue le jour avec pour but de faciliter l'interop\'erabilit\'e entre NUI et les interfaces utilisateurs organiques (OUI) pour les p\'eriph\'eriques d'interaction naturels comme la Kinect. La Framework OpenNI a offert aux d\'eveloppeurs des interfaces de programmation (API) open source et sont largement consid\'er\'es comme la r\'ef\'erence pour l'utilisation des p\'eriph\'eriques comme la Kinect. Ces derniers permettent d'utiliser les fonctions de reconnaissance vocale et gestuelle et aussi de capture des mouvements du corps.
M\^eme si cette organisation a \'enorm\'ement aid\'ee le monde open-source, elle fut finalement acquise par Apple et, le 23 Avril 2014, a \'et\'e ferm\'ee pour de bon. Point positif : Puisque le SDK (Software Development Kit) OpenNI est open source, la communaut\'e de d\'eveloppeurs peut encore l'utiliser, m\^eme si il n'est plus t\'el\'echargeable.}
\textcolor[rgb]{0.55,0,0}{\Large OpenKinect}

	\begin{center}
			\includegraphics[width=80mm]{OpenKinect.png}\\
	\end{center}

\parag{Bien que l'utilisation de la Kinect f\^ut initialement limit\'ee \`a l'exploitation dans les jeux vid\'eo pour la ligne Xbox, les communaut\'es de d\'eveloppeurs du logiciel libre n'ont pas manqu\'ees de cr\'eativit\'e et de g\'enie pour la lib\'erer et lui offrir moult usages impromptus mais tr\`es bien re\c{c}us de la leur ing\'eniosit\'e. L'une des communaut\'es les plus actives est OpenKinect, elle s'occupe non seulement de d\'evelopper des librairies gratuites et libres pour permettre au p\'eriph\'erique d'\^etre utilis\'ee sur Linux et Mac OS en plus de Microsoft Windows, mais aussi de concevoir un foisonnement d'applications et de programmes mettant sa nouvelle technologie \`a l'\oe uvre !}

\parag{Vous voulez vous joindre \`a cette communaut\'e et commencer \`a d\'evelopper pour la Kinect ? Rien de plus facile, il suffit de visiter le site du projet OpenKinect, \href{http://www.openkinect.org} et suivre les instructions pour installer les librairies et pilotes libFreenect sur votre ordinateur. Une fois cela est chose faite, il n'y aura que votre imagination qui pourra vous limiter !}

\NewsAuthorEx{\small Ghaith}{LIMAM}{
\vspace*{-6mm}
\begin{flushright}
\scriptsize \textcolor{GreenTea}{\Estatically~}Membre, Freeways ISI\\
\end{flushright}
}
