%\FreeChap{picture 840x390}{Name of chap.}
\FreeChap{Kinectmedium.jpeg}{Kinect et Open Source: \normalsize Quel avenir ensemble?}
%\FreeChap{KinectforWind.jpeg}{Kinect et Open Source: \normalsize Quel avenir ensemble?}
%\FreeChap{D-kinect10.jpg}{Kinect et Open Source: \small Quel avenir ensemble?}

%\NewsItem{\textcolor[rgb]{0.55,0,0}{\huge Kinect et Open Source: Quel avenir ensemble?}}

\vspace*{-60mm}


\parag{\hspace*{6mm} En évoquant le mot high-tech, il est impossible de ne pas parler du périphérique Kinect (Projet Natal lors de la phase de développement) et de sa technologie novatrice. Sortie le 4 Novembre 2010 (date de sortie aux ètat-Unis) par Microsoft pour sa console Xbox 360 puis ensuite pour la Xbox One et Windows, Kinect est un périphérique de détection du mouvement et de reconnaissance vocale et faciale permettant à l'utilisateur d'interagir avec la console ou l'ordinateur sans avoir recours à un contrôleur et cela uniquement avec des gestes et des commandes vocales, à travers ce qu'on appelle une Natural User Interface (NUI), ou interface utilisateur naturelle.}

\parag{Ce concept apparaît extrêmement prometteur, et c'est exactement ce que plusieurs personnes se sont dit : le mois même du lancement du gadget pour la Xbox 360, Adafruit Industries a lancé un concours pour la création d'un pilote open-source pour la Kinect dont le vainqueur, annoncé le 10 Novembre, a été Héctor Mart\`in Cantero, un hacker qui a pu produire un pilote Linux capable de capturer le flux vidéo RGB de la caméra du périphérique, mais aussi le flux monochrome généré par la profondeur de champ, après avoir pénétrer les protections matérielles et logicielles mises en place par Microsoft. Suite à ce succès et encore une fois le même mois, un ingénieur de Google dénommé Matt Cutts a lui aussi organisé et financé un concours pour le développement des applications pour Kinect pour rendre cette dernière plus compatible avec le système d'exploitation Linux. Il faut dire qu'ils ne perdent pas de temps ces gens-là !}
\textcolor[rgb]{0.55,0,0}{OpenNI}

\parag{Le monde open source a accueilli à bras ouverts ces initiative, et une organisation nommée OpenNI a presque vue le jour avec pour but de faciliter l'interopérabilité entre NUI et les interfaces utilisateurs organiques (OUI) pour les périphériques d'interaction naturels comme la Kinect. La Framework OpenNI a offert aux développeurs des interfaces de programmation (API) open source et sont largement considérés comme la référence pour l'utilisation des périphériques comme la Kinect. Ces derniers permettent d'utiliser les fonctions de reconnaissance vocale et gestuelle et aussi de capture des mouvements du corps.
Même si cette organisation a énormément aidée le monde open-source, elle fut finalement acquise par Apple et, le 23 Avril 2014, a été fermée pour de bon. Point positif : Puisque le SDK (Software Development Kit) OpenNI est open source, la communauté de développeurs peut encore l'utiliser, même si il n'est plus téléchargeable.}
\textcolor[rgb]{0.55,0,0}{\Large OpenKinect}

	\begin{center}
			\includegraphics[width=80mm]{OpenKinect.png}\\
	\end{center}

\parag{Bien que l'utilisation de la Kinect fût initialement limitée à l'exploitation dans les jeux vidéo pour la ligne Xbox, les communautés de développeurs du logiciel libre n'ont pas manquées de créativité et de génie pour la libérer et lui offrir moult usages impromptus mais très bien reçus de la leur ingéniosité. L'une des communautés les plus actives est OpenKinect, elle s'occupe non seulement de développer des librairies gratuites et libres pour permettre au périphérique d'être utilisée sur Linux et Mac OS en plus de Microsoft Windows, mais aussi de concevoir un foisonnement d'applications et de programmes mettant sa nouvelle technologie à l'\oe uvre !}

\parag{Vous voulez vous joindre à cette communauté et commencer à développer pour la Kinect ? Rien de plus facile, il suffit de visiter le site du projet OpenKinect, \href{http://www.openkinect.org} et suivre les instructions pour installer les librairies et pilotes libFreenect sur votre ordinateur. Une fois cela est chose faite, il n'y aura que votre imagination qui pourra vous limiter !}

\NewsAuthorEx{\small Ghaith}{LIMAM}{
\vspace*{-6mm}
\begin{flushright}
\scriptsize \textcolor{GreenTea}{\Estatically~}Membre, Freeways ISI\\
\end{flushright}
}
